

\chapter{Introduction}
\label{chapter:introduction}


\section{Motivation}


\section{IRIS-Lab Context}

The \ac{ua}'s \ac{iris} conducts research projects using autonomous mobile robots, which communicate through a Wi-Fi network. Currently, this network is confined to the premises of the \ac{iris}, preventing the robots from operating in the remaining \ac{ua}'s buildings. Although the \ac{ua}'s Wi-Fi infrastructure covers most of its edifices, which can be used by the robots, due to security mechanisms, this network proves to be highly restraining, not allowing \ac{p2p} communications through the \ac{ros} - the operating system the robots run on - middleware without additional network equipments. Moreover, these constraints keep developers from being able to interact with the robots through their personal machines, which, if otherwise possible, would be of great interest.

\section{Objectives}

The main goal of this dissertation is to implement a private overlay network manager to be used exclusively by \ac{ua}'s clients. The concept of a manager entails both the definition of a network's client universe (which nodes should be allowed to connect to a certain network) and its respective identification and authentication mechanisms.

In the \ac{iris} scenario, the management platform should provide operations to achieve communication between a team of robots, regardless of their physical location within the campus. Moreover, the authentication and connection to a desired overlay network by the robots must be a seemingless operation, requiring little to no manual configuration.

Finally, all traffic must be encrypted and properly authenticated, to ensure the privacy of the communication.


